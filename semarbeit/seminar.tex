\documentclass[12pt,a4paper]{article}
\usepackage[utf8]{inputenc}
\usepackage[T1]{fontenc}
\usepackage[ngerman]{babel}
\usepackage{geometry}
\usepackage{graphicx}
\usepackage{setspace}
\usepackage{enumitem}
\usepackage{hyperref}
\usepackage{csquotes}
\usepackage{natbib}
\geometry{margin=2.5cm}
\setstretch{1.2}

\title{Smart Beach Resort – IoT-basiertes Ferienhausmanagement}
\author{Gruppe B -- IoT Seminararbeit}
\date{2025}

\begin{document}
\maketitle
\tableofcontents
\newpage

\section{Einleitung}
Das Projekt \textit{Smart Beach Resort} zeigt, wie IoT-Technologien ein vernetztes Ferienhaus-Ökosystem bilden können.
Drei intelligente Ferienhäuser an der niederländischen Nordseeküste werden über Sensorik und Cloud-Kommunikation gesteuert.
Besondere Herausforderungen der Umgebung sind hohe Luftfeuchtigkeit, Salz, Sand und Sturm.

Ziel ist es, drei Einzel-Use-Cases zu entwickeln und zu einem Gesamt-Use-Case zu integrieren.
Die Lösung adressiert Betreiber, Facility-Manager und Ferienhausvermieter, die ein skalierbares, wartungsarmes System benötigen.

Mehrwert:
\begin{itemize}[noitemsep]
  \item Automatisierte Sturm- und Sicherheitswarnungen
  \item Energieeffizientes Gebäudemanagement
  \item Digitale Check-in/Check-out-Abläufe
  \item Geringe Betriebskosten durch Fernsteuerung
\end{itemize}

\newpage
\section{Use Case View}

\subsection{Einzel-Use Cases}
\textbf{Haus 1 – Wetter \& Sturm-Überwachung}\\
Sensoren: Temperatur-, Feuchte-, Dampf-, Gas-, Bewegungsmelder.\\
Aktoren: Buzzer, LCD-Display.\\
Funktion: Frühwarnung bei Sturm oder Gas, Korrosionsschutz, Sicherheitsalarm.

\vspace{0.5em}
\textbf{Haus 2 – Gäste-Komfort \& Sicherheit}\\
Sensorik/Aktorik: RFID, Buttons, LCD, RGB-LED, Servo, Yellow-LED.\\
Funktion: kontaktloser Zugang, Service-Anfragen, automatische Rollos, Beleuchtungssteuerung.

\vspace{0.5em}
\textbf{Haus 3 – Energie \& Wartung}\\
Sensorik/Aktorik: Motor/Lüfter, Gas-Sensor, Buzzer, LCD, Button.\\
Funktion: Lüftungssteuerung, Wartungsalarm, Reinigungssignalisierung.

\subsection{Gesamt-Use Case – Smart Beach Resort Management}
Alle drei Häuser bilden ein gemeinsames IoT-System:
\begin{itemize}[noitemsep]
  \item Zentrales Dashboard (Node-RED) zur Gesamtsteuerung
  \item Automatische Fenster- und Lüftungssteuerung bei Sturm
  \item Synchronisierte Reinigungs- und Energieprozesse
  \item Bluetooth für lokale Haus-zu-Haus-Kommunikation
  \item Internet-Anbindung über Handy-Hotspots
\end{itemize}

\begin{quote}
Abbildung 1: Use-Case-Diagramm mit Akteuren (Gast, Host, Reinigungspersonal)
\end{quote}

\newpage
\section{Logical View}
Die logische Architektur besteht aus vier Schichten:
\begin{enumerate}[noitemsep]
  \item \textbf{Device Layer:} ESP32-Controller mit Sensor- und Aktorschnittstellen.
  \item \textbf{Communication Layer:} MQTT über HiveMQ Cloud.
  \item \textbf{Logic Layer:} Node-RED-Flows für Ereignisverarbeitung.
  \item \textbf{Application Layer:} Dashboard und mobile App.
\end{enumerate}

\textbf{Kernkomponenten:}
\begin{itemize}[noitemsep]
  \item \texttt{SensorManager} – Erfassung und Kalibrierung
  \item \texttt{MQTTClient} – Verbindungsaufbau, Publish/Subscribe
  \item \texttt{RuleEngine} – Regelbasierte Logik (z. B. Sturm)
  \item \texttt{DisplayManager} – Statusausgabe
  \item \texttt{ServiceInterface} – Kommunikation mit App
\end{itemize}

\begin{quote}
Abbildung 2: Klassendiagramm mit MQTT-Themen und Zustandsbeziehungen
\end{quote}

\section{Development View}
Das System ist modular entwickelt.
Jede Entwickler*in implementiert ein Haus-Modul mit identischer Kommunikationsschnittstelle.

\textbf{Software-Management:}
\begin{itemize}[noitemsep]
  \item Versionskontrolle über GitHub
  \item Arduino IDE 2.x für ESP32
  \item Node-RED-Flows als JSON-Dateien
\end{itemize}

\begin{verbatim}
/house1/firmware/
/house2/firmware/
/house3/firmware/
/node-red/flows.json
/docs/architecture/
\end{verbatim}

\textbf{Kommunikation:}
MQTT-Topics wie \texttt{resort/house1/telemetry/temp} oder
\texttt{resort/all/cmd/storm} steuern alle Interaktionen.
QoS 1 für sicherheitsrelevante Daten, Retained Status für Systemzustände.

\begin{quote}
Abbildung 3: Komponentendiagramm (3 ESP32 + Node-RED + HiveMQ)
\end{quote}

\newpage
\section{Process View}
\subsection{Beispielprozess – Sturmwarnung}
\begin{enumerate}[noitemsep]
  \item Haus 1 erkennt erhöhte Feuchtigkeit und Temperaturabfall.
  \item MQTT-Nachricht \texttt{resort/all/cmd/storm\_alert=true}.
  \item Node-RED sendet Befehle:
  \begin{itemize}
    \item \texttt{resort/house2/cmd/shutters=DOWN}
    \item \texttt{resort/house3/cmd/ventilation=OFF}
  \end{itemize}
  \item Häuser bestätigen Status über \texttt{resort/{house}/status}.
\end{enumerate}

\subsection{Kommunikationsparameter}
\begin{itemize}[noitemsep]
  \item Keep-Alive 60 s, Heartbeat 30 s  
  \item QoS 1 für kritische Events  
  \item Nachrichtengröße ≈ 150 Byte  
  \item Datenrate ≈ 6–12 kB pro Sensor und Stunde
\end{itemize}

\subsection{Beispielprozess – Gästewechsel}
\begin{enumerate}[noitemsep]
  \item RFID-Check-out → Event \texttt{checkout}.
  \item Node-RED startet Reinigungstimer für Haus 3.
  \item Reinigung meldet Button-Event \texttt{cleaned}.
  \item Node-RED reaktiviert Lüftung und bereitet Check-in vor.
\end{enumerate}

\begin{quote}
Abbildung 4: Sequenzdiagramm für Sturm- und Check-out-Prozesse
\end{quote}

\newpage
\section{Physical View}
\textbf{Hardware-Mapping:}

\begin{tabular}{p{4cm}p{10cm}}
\textbf{Komponente} & \textbf{Beschreibung} \\
ESP32 DevBoard & Controller mit WiFi und Bluetooth \\
Sensoren & DHT11, Steam, Gas, PIR, RFID, Buttons \\
Aktoren & Buzzer, Servo, LED, Lüfter \\
Gateway & Smartphone-Hotspot pro Haus \\
Broker & HiveMQ Cloud (Serverless) \\
Controller & Node-RED auf privatem Server \\
\end{tabular}

\vspace{1em}
Netzstruktur:
\begin{itemize}[noitemsep]
  \item Jedes Haus → eigenes IoT-Node mit Hotspot-Anbindung.
  \item Kommunikation über MQTT zu HiveMQ Cloud.
  \item Node-RED abonniert \texttt{resort/\#} für zentrale Steuerung.
\end{itemize}

\begin{quote}
Abbildung 5: Deployment-Diagramm (ESP32 – HiveMQ – Node-RED – App)
\end{quote}

\section{Schlussfolgerungen}
Das Projekt zeigt, dass ein verteiltes IoT-System mit mobilen Hotspots zuverlässig funktioniert, wenn ein Cloud-basiertes Kommunikationsframework eingesetzt wird.  
HiveMQ + Node-RED bietet eine modulare, skalierbare Architektur.

\textbf{Ergebnisse:}
\begin{itemize}[noitemsep]
  \item Sichere, stabile Kommunikation über MQTT Cloud-Broker
  \item Automatisierte Prozesssteuerung per Node-RED
  \item Cloud- und On-Premise-Betrieb möglich
  \item Dezentrale Entwicklung ohne physische Vernetzung
\end{itemize}

\textbf{Kosten:}
\begin{itemize}[noitemsep]
  \item Hardware: ≈ 210 € (3 × KS5009-Kits)
  \item Cloud-Infrastruktur: 0–10 €/Monat
\end{itemize}

\textbf{Key Take-Aways:}
\begin{itemize}[noitemsep]
  \item MQTT ist das stabilste Kommunikationsprotokoll für IoT.
  \item Cloud-Broker beseitigen NAT-Probleme bei Hotspots.
  \item Node-RED vereinfacht Regellogik und Visualisierung.
  \item System ist erweiterbar und demofähig unter realen Bedingungen.
\end{itemize}

\newpage
\bibliographystyle{apalike}
\bibliography{references}

\end{document}
