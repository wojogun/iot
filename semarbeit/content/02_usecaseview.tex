\section{Use Case View}

\subsection{Einzel-Use Cases}
\textbf{Haus 1 – Wetter \& Sturm-Überwachung}\\
Sensoren: Temperatur-, Feuchte-, Dampf-, Gas-, Bewegungsmelder.\\
Aktoren: Buzzer, LCD-Display.\\
Funktion: Frühwarnung bei Sturm oder Gas, Korrosionsschutz, Sicherheitsalarm.

\vspace{0.5em}
\textbf{Haus 2 – Gäste-Komfort \& Sicherheit}\\
Sensorik/Aktorik: RFID, Buttons, LCD, RGB-LED, Servo, Yellow-LED.\\
Funktion: kontaktloser Zugang, Service-Anfragen, automatische Rollos, Beleuchtungssteuerung.

\vspace{0.5em}
\textbf{Haus 3 – Energie \& Wartung}\\
Sensorik/Aktorik: Motor/Lüfter, Gas-Sensor, Buzzer, LCD, Button.\\
Funktion: Lüftungssteuerung, Wartungsalarm, Reinigungssignalisierung.

\subsection{Gesamt-Use Case – Smart Beach Resort Management}
Alle drei Häuser bilden ein gemeinsames IoT-System:
\begin{itemize}[noitemsep]
	\item Zentrales Dashboard (Node-RED) zur Gesamtsteuerung
	\item Automatische Fenster- und Lüftungssteuerung bei Sturm
	\item Synchronisierte Reinigungs- und Energieprozesse
	\item Bluetooth für lokale Haus-zu-Haus-Kommunikation
	\item Internet-Anbindung über Handy-Hotspots
\end{itemize}

\begin{figure}[h!]
	\centering
	\includegraphics[width=0.8\textwidth]{img/tbd.jpg}
	\caption{Use-Case-Diagramm mit Akteuren (Gast, Host, Reinigungspersonal)}
	\label{fig:systemuebersicht}
\end{figure}



