\section{Process View}
\subsection{Beispielprozess – Sturmwarnung}
\begin{enumerate}[noitemsep]
	\item Haus 1 erkennt erhöhte Feuchtigkeit und Temperaturabfall.
	\item MQTT-Nachricht \texttt{resort/all/cmd/storm\_alert=true}.
	\item Node-RED sendet Befehle:
	\begin{itemize}
		\item \texttt{resort/house2/cmd/shutters=DOWN}
		\item \texttt{resort/house3/cmd/ventilation=OFF}
	\end{itemize}
	\item Häuser bestätigen Status über \texttt{resort/{house}/status}.
\end{enumerate}

\subsection{Kommunikationsparameter}
\begin{itemize}[noitemsep]
	\item Keep-Alive 60 s, Heartbeat 30 s  
	\item QoS 1 für kritische Events  
	\item Nachrichtengröße ca. 150 Byte  
	\item Datenrate ca. 6–12 kB pro Sensor und Stunde
\end{itemize}

\subsection{Beispielprozess – Gästewechsel}
\begin{enumerate}[noitemsep]
	\item RFID-Check-out → Event \texttt{checkout}.
	\item Node-RED startet Reinigungstimer für Haus 3.
	\item Reinigung meldet Button-Event \texttt{cleaned}.
	\item Node-RED reaktiviert Lüftung und bereitet Check-in vor.
\end{enumerate}

\begin{figure}[h!]
	\centering
	\includegraphics[width=0.8\textwidth]{img/tbd.jpg}
	\caption{Sequenzdiagramm für Sturm- und Check-out-Prozesse}
	\label{fig:systemuebersicht}
\end{figure}

