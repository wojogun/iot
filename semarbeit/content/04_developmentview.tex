\section{Development View}
Das System ist modular entwickelt.
Jede Entwickler*in implementiert ein Haus-Modul mit identischer Kommunikationsschnittstelle.

\textbf{Software-Management:}
\begin{itemize}[noitemsep]
	\item Versionskontrolle über GitHub
	\item Arduino IDE 2.x für ESP32
	\item Node-RED-Flows als JSON-Dateien
\end{itemize}

\begin{verbatim}
	/house1/firmware/
	/house2/firmware/
	/house3/firmware/
	/node-red/flows.json
	/docs/architecture/
\end{verbatim}

\textbf{Kommunikation:}
MQTT-Topics wie \texttt{resort/house1/telemetry/temp} oder
\texttt{resort/all/cmd/storm} steuern alle Interaktionen.
QoS 1 für sicherheitsrelevante Daten, Retained Status für Systemzustände.
\begin{figure}[h!]
	\centering
	\includegraphics[width=0.8\textwidth]{img/tbd.jpg}
	\caption{Komponentendiagramm (3 ESP32 + Node-RED + HiveMQ)}
	\label{fig:systemuebersicht}
\end{figure}
