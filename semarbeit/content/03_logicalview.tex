\section{Logical View}
Die logische Architektur besteht aus vier Schichten:
\begin{enumerate}[noitemsep]
	\item \textbf{Device Layer:} ESP32-Controller mit Sensor- und Aktorschnittstellen.
	\item \textbf{Communication Layer:} MQTT über HiveMQ Cloud.
	\item \textbf{Logic Layer:} Node-RED-Flows für Ereignisverarbeitung.
	\item \textbf{Application Layer:} Dashboard und mobile App.
\end{enumerate}

\textbf{Kernkomponenten:}
\begin{itemize}[noitemsep]
	\item \texttt{SensorManager} – Erfassung und Kalibrierung
	\item \texttt{MQTTClient} – Verbindungsaufbau, Publish/Subscribe
	\item \texttt{RuleEngine} – Regelbasierte Logik (z. B. Sturm)
	\item \texttt{DisplayManager} – Statusausgabe
	\item \texttt{ServiceInterface} – Kommunikation mit App
\end{itemize}

\begin{figure}[h!]
	\centering
	\includegraphics[width=0.8\textwidth]{img/tbd.jpg}
	\caption{Abbildung 2: Klassendiagramm mit MQTT-Themen und Zustandsbeziehungen}
	\label{fig:systemuebersicht}
\end{figure}



