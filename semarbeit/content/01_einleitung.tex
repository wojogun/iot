\section{Einleitung}
Das Projekt \textit{Smart Beach Resort} untersucht, wie moderne IoT-Technologien in einem praxisnahen Umfeld einsetzbar sind, um den Betrieb von Ferienhäusern effizienter, sicherer und nachhaltiger zu gestalten. 
An der niederländischen Nordseeküste stehen drei intelligente Modellhäuser, die gemeinsam ein digitales Resort bilden. 
Die Häuser sind durch ein hybrides Kommunikationssystem miteinander verbunden, das lokale Bluetooth-Verbindungen mit einer Cloud-basierten MQTT-Infrastruktur kombiniert. 
Jedes Haus erfüllt eine spezifische Funktion und trägt zum Gesamtsystem bei: Überwachung von Wetter und Sturm, Gäste-Komfort und Sicherheit sowie Energie- und Wartungsmanagement. 
Die Verbindung dieser Einzelfunktionen schafft ein autonom agierendes System, das auf Umweltbedingungen und Nutzeraktionen selbstständig reagiert.

Die physische Umgebung stellt besondere technische Herausforderungen dar. 
Salzhaltige Luft, hohe Luftfeuchtigkeit, Sandablagerungen und häufige Stürme führen zu Korrosionsgefahr, Sensorbelastung und instabilen Funkbedingungen. 
Ziel des Projekts ist daher nicht nur die Entwicklung funktionaler Einzellösungen, sondern auch der Nachweis, dass ein IoT-System unter realistischen Umweltbedingungen zuverlässig arbeiten kann. 
Darüber hinaus soll gezeigt werden, wie drei getrennt entwickelte Einheiten zu einem konsistenten Gesamtsystem verknüpft werden können, das sowohl lokal als auch cloudbasiert funktioniert.

Der Ansatz basiert auf der Nutzung standardisierter IoT-Komponenten. 
Die Häuser sind mit einem \textbf{ESP32}-Controller ausgestattet, der Sensordaten über Temperatur, Luftfeuchtigkeit, Bewegung, Gasgehalt oder RFID-basierte Zugänge erfasst. 
Die Daten werden über das \textbf{MQTT-Protokoll} an den Cloud-Broker \textbf{HiveMQ} übertragen und dort verarbeitet. 
Ein zentraler \textbf{Node-RED}-Server übernimmt die Steuerungslogik, aggregiert Ereignisse und reagiert auf erkannte Zustände, beispielsweise durch das Schließen von Fenstern bei Sturm oder das Abschalten der Lüftung bei hoher Feuchtigkeit. 
Damit entsteht eine klare funktionale Trennung zwischen Geräte-, Kommunikations- und Steuerungsebene.

Die Entwicklung erfolgt dezentral. 
Jede Teilgruppe arbeitet an einem eigenen Hausmodul, das unabhängig funktionsfähig ist, sich jedoch über ein gemeinsames Topic-System in die Resort-Architektur integrieren lässt. 
Da die Entwickler an unterschiedlichen Standorten arbeiten, wird die Verbindung über mobile Hotspots realisiert. 
Dies zeigt, dass auch verteilte Entwicklungsteams IoT-Systeme aufbauen können, ohne auf lokale Netzwerke oder physische Kopplung angewiesen zu sein. 
Die Cloud-Anbindung erlaubt die Zusammenarbeit in Echtzeit und vereinfacht die Integration der einzelnen Module zu einem gemeinsamen Use Case.

Das Projekt richtet sich an Betreiber von Ferienanlagen, kleine Hotels und Gebäudeverwalter, die ihre Objekte aus der Ferne überwachen und steuern möchten. 
Die Kombination aus kostengünstiger Mikrocontroller-Hardware, Cloud-Integration und visueller Steuerungslogik (Node-RED) zeigt, dass sich ein voll funktionsfähiges Smart-Home-System auch ohne teure Speziallösungen realisieren lässt. 
Der Lösungsansatz kann auf andere Einsatzgebiete wie Studentenwohnheime, Serviced Apartments oder modulare Ferienparks übertragen werden. 
Wirtschaftlich betrachtet bietet das System eine Reduktion von Personalaufwand und Energieverbrauch sowie eine Erhöhung der Betriebssicherheit und Gästezufriedenheit.

\begin{quote}
	\textbf{Mehrwert:}
	\begin{itemize}[noitemsep]
		\item Automatisierte Sturm- und Sicherheitswarnungen für das gesamte Resort
		\item Energieeffizientes Gebäudemanagement durch bedarfsorientierte Steuerung
		\item Digitale Check-in/Check-out-Prozesse über RFID-Zugangssysteme
		\item Vereinfachte Wartungskoordination und Zustandsüberwachung in Echtzeit
		\item Skalierbarkeit auf größere Ferienanlagen durch Cloud-Integration
	\end{itemize}
\end{quote}

Insgesamt zeigt das \textit{Smart Beach Resort}-Projekt, dass ein durchdachtes Zusammenspiel von Sensorik, Cloud-Infrastruktur und Automatisierungslogik eine robuste IoT-Architektur ermöglicht, die sowohl für die Forschung als auch für den praktischen Einsatz in der Gebäudeverwaltung relevant ist. 
Es dient als Blaupause für verteilte Smart-Home-Systeme und demonstriert, dass Cloud-fähige IoT-Frameworks wie HiveMQ und Node-RED die Grundlage für zukunftsfähige, modulare und wartbare Smart-Environment-Lösungen bilden.
