\section{Schlussfolgerungen}
Das Projekt zeigt, dass ein verteiltes IoT-System mit mobilen Hotspots zuverlässig funktioniert, wenn ein Cloud-basiertes Kommunikationsframework eingesetzt wird.  
HiveMQ + Node-RED bietet eine modulare, skalierbare Architektur.

\textbf{Ergebnisse:}
\begin{itemize}[noitemsep]
	\item Sichere, stabile Kommunikation über MQTT Cloud-Broker
	\item Automatisierte Prozesssteuerung per Node-RED
	\item Cloud- und On-Premise-Betrieb möglich
	\item Dezentrale Entwicklung ohne physische Vernetzung
\end{itemize}

\textbf{Kosten:}
\begin{itemize}[noitemsep]
	\item Hardware: Anschaffung ca. 210 € (3 × KS5009-Kits)
	\item Cloud-Infrastruktur: 0 €/Monat bzw. Serverkosten bei Neuanmietung
\end{itemize}

\textbf{Key Take-Aways:}
\begin{itemize}[noitemsep]
	\item MQTT ist das stabilste Kommunikationsprotokoll für IoT.
	\item Cloud-Broker beseitigen NAT-Probleme bei Hotspots.
	\item Node-RED vereinfacht Regellogik und Visualisierung.
	\item System ist erweiterbar und demofähig unter realen Bedingungen.
\end{itemize}